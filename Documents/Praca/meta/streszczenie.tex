\chapter*{Streszczenie (Karolina Makuch)}
 \par Tematem niniejszej pracy inżynierskiej jest platforma do planowania podróży z klientem mobilnym. W tym celu skorzystano z ogólnodostępnych baz danych zawierających potrzebne informacje związane z poszczególnymi atrakcjami turystycznymi oraz możliwościami transportu pomiędzy dwoma lokalizacjami. Dodano opcję przechowywania istotnych dokumentów w formie skanów. Zaimplementowano również mechanizm rekomendacji pojedynczych elementów podróży. Umożliwiono udostępnianie całego planu innym użytkownikom aplikacji oraz jego elementu w medium społecznościowym.
 
 \par Rozdział 1 wprowadza do tematu pracy  oraz uzasadnia cele stworzenia aplikacji mobilnej. 
 
 \par W rozdziale 2 zamieszczono zestawienie podobnych rozwiązań definiując kluczowe wady i zalety każdej z nich.

\par Rozdział 3 wyróżnia najważniejsze wymagania oraz cele poprzez przedstawienie esencji specyfikacji wymagań systemowych.

\par W rozdziale 4 omawiany jest projekt systemu aplikacji z podziałem na podsystemy po stronie serwera oraz po stronie aplikacji. Zawiera również przykładowe scenariusze użycia.

\par Rozdział 5 uzasadnia fundamentalne decyzje projektowe związane z wyborem języka programowania, środowisk programistycznych oraz systemu kontroli wersji.

\par W rozdziale 6 opisano zastosowany wzorzec projektowy aplikacji mobilnej. Przedstawiono metody zaimplementowania konkretnych funkcjonalności oraz rozwiązanie problemu komunikacji z aplikacją serwerową. 

\par Rozdział 7 przybliża podejście do implementacji serwera, uwzględniając połączenie z bazą danych oraz z zewnętrznym API. Obrazuje również proces rekomendacji danych miejsc użytkownikowi.

\par W rozdziale 8 zamieszczono opis pracy z wybraną bazą danych oraz diagram związków encji.

\par Rozdział 9 dotyczy rodzajów przeprowadzonych testów oraz przedstawia poszczególne z nich.

\par W rozdziale 10 zobrazowano najważniejsze funkcjonalności poprzez wyróżnienie poszczególnych etapów czynności oraz przedstawienie zrzutów ekranu. 

\par Rozdział 11 podsumowuje całokształt prac.

\par \textbf{Podział prac} (dokładniejsza rozpiska w dodatku A):
\begin{itemize}
\item Anna Malizjusz --  projekt i implementacja bazowej postaci serwera, komunikacja z zewnętrznym API oraz implementacja wyszukiwania poszczególnych obiektów w aplikacji
\item Dorota Tomczak -- projekt i implementacja bazowej postaci aplikacji mobilnej, mechanizm rekomendacji oraz skanowania biletów
\item Magdalena Solecka -- projekt i implementacja początkowej postaci bazy danych, mechanizm dodawania podróży oraz implementacja trybu usuwania
\item Karolina Makuch -- wyszukiwanie i dodawanie użytkowników do znajomych, mechanizmy udostępniania oraz manualna realizacja planu
\end{itemize}
\newline
\newline\textbf{Słowa kluczowe}: aplikacja mobilna, podróż, planowanie
\newline\textbf{Dziedzina nauki i techniki, zgodnie z wymogami OECD}: Sprzęt komputerowy i architektura komputerów
