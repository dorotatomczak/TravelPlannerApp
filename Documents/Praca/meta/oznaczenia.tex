\chapter*{Wykaz najważniejszych oznaczeń i skrótów}
\addcontentsline{toc}{chapter}{Wykaz najważniejszych oznaczeń i skrótów}
\noindent REST (Representational State Transfer) -- transfer stanu przez reprezentację \\
JSON (JavaScript Object Notation) -- obiektowy sposób zapisu danych \\
BJSON (binary JSON) -- JSON kodowany binarnie \\
JWT (JSON Web Token) -- token w formacie JSON używany w celu zabezpieczenia dostępu aplikacji \\
API (Application Programming Interface) -- interfejs aplikacji dostępny dla programisty \\
MVP (Model View Presenter) -- wzorzec architektoniczny\\
IDE (Integrated Development Environment) -- zintegrowane środowisko programistyczne\\
DTO (Data State Object) -- wzorzec projektowy, rodzaj kontenera na dane\\
CRUD (create, read, update, delete) -- zestaw czterech podstawowych funkcji dostępu do bazy danych \\
DAO (Data access object) -- klasa będąca odzwierciedleniem tabeli bazodanowej \\
SQL (Standard Query Language) -- język zapytań do niektórych baz danych (w tym PostgreSQL) \\
JDBC (Java Database Connectivity) -- zestaw funkcji umożliwiających połączenie z bazą danych PostgreSQL \\
ERD (Entity Relation Diagram) -- diagram relacyjnej bazy danych \\
IDE (Integrated Development Environment) -- zintegrowane środowisko programistyczne \\