\chapter{Wstęp i cel pracy (Karolina Makuch)}
\par Aplikacje mobilne są coraz popularniejszym rozwiązaniem stosowanym w różnych dziedzinach. Rozwój technologiczny potęguje możliwości telefonów przez co coraz więcej firm decyduje się na przenoszenie swoich aplikacji na platformy mobilne. Kilkanaście lat temu  służyły one głównie do komunikacji. Obecnie szeroka gama ich funkcjonalności zaczyna dorównywać ich \textit{większym braciom} -- komputerom.
\par Znaczącą przewagę zyskują dzięki swoim rozmiarom. Są znacznie wygodniejsze w codziennej obsłudze. Większość z nich zmieści się do kieszeni swojego właściciela. Dzięki temu coraz częściej są wykorzystywane w miejscach publicznych, nie zabierając przestrzeni innym osobom. Umożliwiają między innymi ustalanie szczegółów poprzez pisanie wiadomości mailowych, wyszukiwanie potrzebnych informacji jak również odpoczynek poprzez zagranie w grę mobilną.
\par Celem pracy było zaprojektowanie i zaimplementowanie aplikacji umożliwiającej zaplanowanie podróży poprzez integrację systemu z dostępnymi bazami danych o atrakcjach turystycznych oraz możliwości transportu pomiędzy dwoma lokalizacjami. Postanowiono również zaimplementować algorytm rekomendujący plan podróży. Zdecydowano się także na dodanie mechanizmu skanowania dokumentów. 
\par Przed przystąpieniem do fazy projektowania opracowano specyfikację wymagań systemowych z uwzględnieniem udziałowców, celów systemu (biznesowych oraz funkcjonalnych), otoczenie systemu (użytkownicy oraz systemy zewnętrzne), przewidywalne komponenty systemu (sprzętowe, programowalne oraz podsystemy), wymagania (funkcjonalne, programowe, na dane, wydajnościowe, dodatkowe, w zakresie wiarygodności, elastyczności oraz satysfakcji), sytuacje wyjątkowe i kryteria akceptacyjne.
\par Zdecydowano się na implementację aplikacji dla systemu mobilnego Android, rozwijanego przez firmę Google. Opiera się on na jądrze systemu operacyjnego Linux oraz na oprogramowaniu posiadającym licencję GNU (\textit{GNU General Public License}).
 \par Od blisko dziesięciu lat smartfony z systemem Android znajdują się na pierwszym miejscu w światowym rankingu najczęściej kupowanych telefonów. Podczas jednego roku (2017) sprzedano blisko 1,3 mln urządzeń z tym systemem. Jest on zdecydowanie bardziej otwarty (dzięki mniejszym kosztom implementacji) oraz dostępny (przystępniejsze ceny urządzeń) niż konkurencja.
\par Potrzebne informacje są przechowywane poza aplikacją. Wykorzystano w tym celu system bazy danych \textit{PostgreSQL}. Zaimplementowano również serwer umożliwiający komunikację pomiędzy bazą a aplikacją oraz z zewnętrznym API.
\par Zdefiniowano również cele do osiągnięcia. Postawiono na zwiększenie zadowolenia użytkownika z podróży poprzez zmniejszenie czasu poświęconego na jej organizację. W tym celu dodano między innymi:

\begin{itemize}

\item funkcjonalności umożliwiające znalezienie potrzebnych informacji na temat poszczególnych atrakcji oraz transportu (dzięki komunikacji z zewnętrznym API),

\item  wyświetlenie rekomendacji na podstawie opinii innych użytkowników (wykorzystując algorytm \textit{Collaborative Filtering}),

\item umożliwienie edycji jednego planu podróży poprzez kilku użytkowników,

\item przechowywanie istotnych dokumentów w postaci skanów – odpowiednio przekształconych zdjęć wykonanych przez użytkownika (wykorzystując algorytmy z biblioteki \textit{OpenCV}),

\end{itemize}

\par Rozdział 2 zawiera przegląd podobnych rozwiązań, które są dostępne na rynku aplikacji mobilnych. Pod uwagę wzięto TripIt: Travel Planner, Google Trips -- Travel Planner, Sygic Travel: Planuj Podróż oraz Expedia. W przypadku każdej z nich zidentyfikowano główne wady i zalety. W rozdziale 3 przedstawiono ekstrakt specyfikacji wymagań systemowych opisujący udziałowców projektu oraz otoczenie systemu. Wyróżniono w nim również kluczowe wymagania oraz cele dla powstałej aplikacji. Rozdział 4 omawia projekt systemu aplikacji, wyróżniając podsystemy po stronie aplikacji mobilnej oraz po stronie serwera. Przygotowano także najistotniejsze scenariusze użycia określając sposoby korzystania z aplikacji przez użytkowników. Rozdział 5 opisuje najważniejsze decyzje projektowe, uzasadniając wybór języka Kotlin oraz opisując zalety wybranych środowisk programistycznych (Android Studio, Intellij IDEA), systemu kontroli wersji (Github) i zdalnego serwera (Heroku). 

\par Rozdział 6 prezentuje projekt wzorca zastosowanego po stronie aplikacji mobilnej (model -- widok -- prezenter), sposób jej komunikacji z aplikacją serwerową oraz przybliża sposób implementacji poszczególnych funkcjonalności. Rozdział 7 zawiera informacje dotyczące implementacji serwera (RESTful Web Service), komunikacji z zewnętrznym API, opis dostępu do bazy danych, obsługi sytuacji wyjątkowych oraz mechanizmu rekomendacji wzorującym się na technice Collaborative Filtering. Rozdział 8 zawiera uzasadnienie wyboru relacyjnej bazy danych (PostgreSQL) oraz opis pracy z nią związanej. Zawiera również schemat ERD (Entity Relationship Diagram). W rozdziale 9 wymieniono zaimplementowane testy jednostkowe, instrumentalne oraz dostępu do bazy danych. Wytłumaczono w nim także zasady ich działania. Rozdział 10 został zatytułowany Podręcznik użytkownika. Składa się z opisów poszczególnych czynności mających na celu skorzystanie z danej funkcjonalności. Oprócz tego prezentuje aktualny interfejs użytkownika.
