\chapter{Specyfikacja wymagań systemowych -- ekstrakt (Anna Malizjusz)}
\par W celu lepszego zrozumienia wymagań projektu inżynierskiego przygotowano dokument SWS (Specyfikacji Wymagań Systemowych) w całości umieszczony na załączonej do pracy płycie~CD. Dokonano identyfikacji udziałowców projektu i otoczenia systemu z uwzględnieniem użytkowników oraz systemów zewnętrznych. Wyróżniono cele projektu oraz wymagania, których spełnienie będzie kluczowe dla końcowej akceptacji systemu. Poniżej przedstawiono najistotniejsze elementy SWS, które miały największy wpływ na projekt i~implementację systemu.
\newline
\newline
\noindent Wyróżniono udziałowców, których wpływ na system powinien być największy:

\begin{itemize}
\item programiści,
\item promotor,
\item użytkownik.
\end{itemize}

\noindent A także dokumenty, które należało uwzględnić w pracy:

\begin{itemize}
\item regulamin aplikacji,
\item rozporządzenie o ochronie danych osobowych (RODO).
\end{itemize}

\par Docelowego użytkownika zidentyfikowano jako młodego człowieka, najczęściej studenta o ograniczonym budżecie i czasie, który może poświęcić na planowanie podróży. Posiada on smartfona z dostępem do internetu oraz używa systemu operacyjnego Android. Chce podróżować i kontaktować się z przyjaciółmi poprzez aplikację. Często zapomina o terminach i~koniecznych dokumentach, więc potrzebuje przypomnień oraz dostępu do skanów przy pomocy telefonu.

\par Wyróżniono systemy zewnętrzne, z którymi zintegrowano aplikację. Skupiono się na dostępie do zewnętrznego API udostępnionego przez serwis Here\cite{Here}, które umożliwiało dostęp do map, nawigacji i wyszukiwania obiektów takich jak hotele, restauracje, zabytki, itp. Zaletą serwisu był darmowy dostęp do danych. Here umożliwiało wykonanie 250 tys. zapytań miesięcznie bez dodatkowych opłat, a każdy kolejny tysiąc kosztował 1\$, co zostało uznane za wystarczające dla testowania aplikacji. Uwzględniono również system GPS, który był niezbędny do zrealizowania podstawowych funkcjonalności, np.~wyszukiwania obiektów w pobliżu aktualnej lokalizacji. W tym celu zdecydowano skorzystać z możliwości oferowanych przez serwis Google Play w bibliotece Gsm Location\cite{gms.location}.

\par
\noindent \newline Określono najważniejsze cele projektu:

\begin{itemize}
\item zwiększenie zadowolenia z podróży,
\item zaspokajanie potrzeb informacyjnych użytkowników,
\item zoptymalizowanie trasy,
\item zmniejszenie ilości spóźnień,
\item ułatwienie komunikacji pomiędzy użytkownikami,
\item ułatwienie możliwości koordynacji planu dnia przez użytkownika,
\item skrócenie czasu oczekiwania na dany środek transportu,
\item zmniejszenie czasu przeznaczonego na planowanie podróży,
\item zmniejszenie poziomu stresu użytkowników podczas planowania podróży.
\end{itemize}

\par
\noindent\newline Zidentyfikowano wymagania funkcjonalne, które jednocześnie stanowiły kryteria akceptacyjne projektu: 

\begin{itemize}
\item rejestracja i logowanie użytkownika,
\item stworzenie, przeglądanie i edycja planu podróży i planu dnia,
\item wyszukanie elementu w pobliżu danej lokalizacji,
\item wyszukanie zakwaterowania,
\item wyszukanie i zaproszenie innego użytkownika do wyświetlania lub edycji podróży,
\item dodanie oceny do planu dnia, podróży lub odwiedzonego miejsca,
\item otrzymanie propozycji na podstawie ocen,
\item wyszukanie transportu między lokalizacjami,
\item wyszukanie najkrótszej trasy,
\item skanowanie biletów i innych dokumentów potrzebnych w trakcie podróży.
\end{itemize}

\par
\noindent Wyróżniono dodatkowe wymagania, które nie były niezbędne do realizacji projektu, ale znacznie zwiększały możliwości aplikacji: 

\begin{itemize}
\item oznaczenie elementu z planu dnia jako wykonany,
\item udostępnianie zrealizowanego punktu planu dnia w mediach społecznościowych,
\item wygenerowanie planu dnia/podróży,
\item zapisywanie podróży i dokumentów na urządzeniu, aby możliwe było korzystanie z nich bez dostępu do internetu,
\item powiadomienie o obiekcie w okolicy,
\item powiadomienie o opóźnieniu,
\item powiadomienia dotyczące lotów,
\item przeglądanie statystyk zrealizowanych podróży.
\end{itemize}

\par Dodatkowo zostały określone wymagania jakościowe dotyczące aplikacji. Ze względu na RODO przetwarzanie danych użytkowników ograniczono do minimum i~zdecydowano o wyświetlaniu użytkownikom informacji o sposobie używania danych. Postanowiono skorzystać z bezpiecznych algorytmów szyfrowania i uwierzytelniania, aby zapewnić danym bezpieczeństwo. 
Zobowiązano się do zapewnienia autentyczności proponowanych podróży, tj. sugerowany czas spędzony w danym obiekcie jest zbliżony do rzeczywistego i trafności polecanych obiektów. Zapytania użytkowników mają być obsługiwane nie dłużej niż 5~s., a układanie planu dnia będzie trwać maksymalnie 10~s.

\par Za docelowe urządzenie przyjęto smartfon z systemem operacyjnym Android, którego minimalna wersja to 5.0 (Lollipop), co miało zapewnić obsługę ponad 94\% urządzeń z systemem Android (dane aktualne na dzień 05.05.2019\cite{Android usage}). Spodziewane wymiary urządzeń to od~115.20~mm x 58.60~mm do~242,8~mm x 189,7~mm. Zaplanowano rozszerzenie działania aplikacji na telefony z systemem operacyjnym iOS.

\par Zwrócono uwagę na zagwarantowanie czytelności interfejsu użytkownika. Zaplanowano użycie stonowanych, niejaskrawych kolorów i ograniczenie dostępnych informacji na jednym ekranie z możliwością przejścia do kolejnych stron lub filtrowania wyników. Dostępne ma być powiększenie ekranu.
