\chapter{Podsumowanie (Anna Malizjusz)}

\par Celem pracy było napisanie aplikacji mobilnej, serwera w architekturze REST oraz przetestowanie obu komponentów. Dodatkowo należało zintegrować aplikację z bazą danych oraz zewnętrznym API dostarczonym przez serwisy \textit{Here} oraz \textit{Google}. Bazowe cele zdefiniowano w dokumencie SWS. 

\par Wynikiem pracy inżynierskiej jest zaimplementowana, działająca w środowisku testowym aplikacja mobilna oraz serwer REST. Zrealizowane zostały wszystkie wymagania o wysokim priorytecie niezbędne do akceptacji projektu. Najważniejsze z nich to:
\begin{itemize}
\item rejestracja i logowanie użytkownika,
\item dodanie, przeglądanie i edycja planu podróży i planu dnia,
\item wyszukanie elementu w pobliżu danej lokalizacji, w szczególności zakwaterowania,
\item wyszukanie innego użytkownika i udostępnienie mu podróży,
\item dodanie oceny do odwiedzonego miejsca,
\item otrzymanie propozycji na podstawie ocen,
\item wyszukanie transportu między lokalizacjami,
\item skanowanie biletów.
\end{itemize}

\par Zaimplementowano też kilka dodatkowych funkcji, takich jak udostępnienie odwiedzonego miejsca w serwisie Facebook oraz oznaczenie planu dnia jako wykonany. Uwzględniano ostrzeżenia o użyciach przestarzałych funkcji, aby w aplikacji korzystano z najnowszych i najbezpieczniejszych praktyk.

\par Aplikację mobilną przetestował zespół programistów oraz kilku testerów. Nie stwierdzono rażących błędów, które uniemożliwiałyby korzystanie z zaimplementowanego rozwiązania. Interfejs okazał się łatwy w obsłudze dla młodych ludzi.

\par W trakcie projektu inżynierskiego zespół doświadczył trudności w implementacji rozwiązania \textit{od zera}, bez bazowego kodu ani dokładnych, zewnętrznych wytycznych. Brakowało również praktyki w prowadzeniu względnie obszernego i skomplikowanego przedsięwzięcia, jakim jest stworzenie działającej aplikacji w pół roku. Wielokrotnie wybrane rozwiązania były uznawane za niewystarczające i zmieniane na lepsze. Początkowo tryb pracy i stosowane konwencje często się zmieniały, jednak w pierwszym etapie projektu udało się ustabilizować wewnętrzne wymagania. Dzięki temu recenzje kodu przebiegały sprawniej.

\par Spotykano się z problemem braku pomocnych informacji w Internecie. Część używanych rozwiązań zostało oznaczonych jako przestarzałe i należało samodzielnie znaleźć aktualniejsze. Czasem część znalezionych informacji była specyficzna dla języka Java i nie istniała banalna konwersja do używanego języka Kotlin.

\par Podczas pracy nad aplikacją mobilną i serwerową zdefiniowano kolejne przydatne funkcjonalności lub ulepszenia istniejących. Postanowiono umożliwić użytkownikowi otrzymywanie powiadomień na adres email, a także umożliwić edycję planu dnia. Dodatkowo należy rozszerzyć aplikację o możliwość zmienienia oraz przypomnienia hasła. W serwisie GitHub dodano nowe zadania odnośnie poprawienia jakości kodu obsługującego zapytania na serwerze, w bazie danych oraz w aplikacji mobilnej. Zdecydowano o przyszłej zmianie niektórych z elementów aplikacji, np. sposobu wyświetlania obiektów na mapie. Jednocześnie planowane jest zaimplementowanie kolejnych funkcjonalności, które zostały oznaczone jako mniej ważne.

\par Zdecydowano o późniejszym dokładniejszym pokryciu kodu testami jednostkowymi, a także o dodaniu większej liczby testów integracyjnych. W celu łatwiejszego i bezpieczniejszego rozwoju aplikacji zostaną zaimplementowane mechanizmy ciągłej integracji (ang. continous integration), które zapobiegną problemom z kompilacją oraz działaniem aplikacji na głównej gałęzi (ang. branch) repozytorium (master).
\par Zaimplementowane rozwiązanie, dzięki zachowaniu dobrych praktyk programistycznych podczas pisania kodu, może być dalej rozwijane. Dotychczasowa praca zespołu zwiększyła jego umiejętności, zarówno programistyczne, jak i zdolność do współpracy z innymi członkami grupy.
