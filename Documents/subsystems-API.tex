\documentclass[10pt,twoside,a4paper]{report}

\usepackage{polski}
\usepackage{geometry}
\usepackage{indentfirst}
\usepackage{sectsty}
\usepackage{helvet}
\usepackage{caption}
\usepackage{multirow}
\usepackage{makecell}
\usepackage{array}
\usepackage{subcaption}
\usepackage{graphicx}

\graphicspath{ {./images/} }

\chapternumberfont{\Large}
\chaptertitlefont{\huge}

\usepackage[font={normalsize,it}]{caption}

\linespread{1.5}
\usepackage[table]{xcolor}
\usepackage{fourier} 
\renewcommand\theadalign{bc}
\renewcommand\theadfont{\bfseries}

\newgeometry{tmargin=3cm, bmargin=3cm, lmargin=2.5cm, rmargin=2.5cm}

\begin{document}

\paragraph{Podsystem zarządzania danymi użytkowników}
\begin{itemize}
\item AddAccount()
\item DeleteAccount()
\item UpdateAccount()
\item AuthenticateUser()
\item FindUser(Name)
\item ClearAccountHistory() - bilety, podróże i oceny

\item AddObjectToFavourites()
\item AddVisitedObject(Category, Rating) - nie wiem, czy to nie będzie w polecającym.
Można by stąd wziąć tylko informacje i niech reszta się dzieje w polecającym.
\item ClearRatingsHistory()

\item AddTicket(Image, Data)
\item UpdateTicket()
\item GetTicket(Key)
\item GetTickets()
\item DeleteTicket(Key)
\item ClearTickets()

\item AddTravel()
\item UpdateTravel()
\item GetTravel(Key)
\item GetTravels()
\item DeleteTravel(Key)
\item ClearTravels()

\item AddFriend()
\item DeleteFriend()
\item GetFriends()
\item FindFriend(Key)
\end{itemize}

\paragraph{Podsystem usług społecznościowych}
\begin{itemize}
\item InviteNewUser(Mail)
\item InviteFriend()
\item DeleteFriend()
\item SendMessage(User, Text/Image)

\item ShareTravel(Mode) - jak w google docs można zaprosić do wyświetlania lub edycji
\item SharePlace() - pokaż znajomemu nowe, ciekawe miejsce
\item ShareOnFacebook/Instagram() - priorytet: niski
\end{itemize}

\paragraph{Podsystem wyszukiwania}
\begin{itemize}
\item FindRoutes()
\item FindTouristObject(Category)
\end{itemize}

\end{document}
