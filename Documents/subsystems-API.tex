\documentclass[10pt,twoside,a4paper]{report}

\usepackage{polski}
\usepackage{geometry}
\usepackage{indentfirst}
\usepackage{sectsty}
\usepackage{helvet}
\usepackage{caption}
\usepackage{multirow}
\usepackage{makecell}
\usepackage{array}
\usepackage{subcaption}
\usepackage{graphicx}

\graphicspath{ {./images/} }

\chapternumberfont{\Large}
\chaptertitlefont{\huge}

\usepackage[font={normalsize,it}]{caption}

\linespread{1.5}
\usepackage[table]{xcolor}
\usepackage{fourier} 
\renewcommand\theadalign{bc}
\renewcommand\theadfont{\bfseries}

\newgeometry{tmargin=3cm, bmargin=3cm, lmargin=2.5cm, rmargin=2.5cm}

\begin{document}

\paragraph{Podsystem zarządzania danymi użytkowników}
\begin{itemize}
\item AddAccount()
\item DeleteAccount()
\item UpdateAccount()
\item AuthenticateUser()
\item FindUser(Name)
\item ClearAccountHistory() - bilety, podróże i oceny

\item AddObjectToFavourites()
\item AddVisitedObject(Category, Rating) - nie wiem, czy to nie będzie w polecającym.
Można by stąd wziąć tylko informacje i niech reszta się dzieje w polecającym.
\item ClearRatingsHistory()

\item AddTicket(Image, Data)
\item UpdateTicket()
\item GetTicket(Key)
\item GetTickets()
\item DeleteTicket(Key)
\item ClearTickets()

\item AddTravel()
\item UpdateTravel()
\item GetTravel(Key)
\item GetTravels()
\item DeleteTravel(Key)
\item ClearTravels()

\item AddFriend()
\item DeleteFriend()
\item GetFriends()
\item FindFriend(Key)
\end{itemize}

\paragraph{Podsystem usług społecznościowych}
\begin{itemize}
\item InviteNewUser(Mail)
\item InviteFriend()
\item DeleteFriend()
\item SendMessage(User, Text/Image)

\item ShareTravel(Mode) - jak w google docs można zaprosić do wyświetlania lub edycji
\item SharePlace() - pokaż znajomemu nowe, ciekawe miejsce
\item ShareOnFacebook/Instagram() - priorytet: niski
\end{itemize}

\paragraph{Podsystem wyszukiwania}
\begin{itemize}
\item FindRoutes()
\item FindTouristObject(Category)
\end{itemize}

\paragraph{Podsystem planowania dnia}
\begin{itemize}
\item addElement()
\item deleteElement()
\item editElement()
\item loadDayPlan()
\item generateDayPlan()
\item deleteDayPlan()
\item addTravel() -nwm czy to tez bo nazwa sugeruje inaczej ale gdzie indziej jak nie tu
\item deleteTravel() - same
\item editTravel() - same
\end{itemize}

\paragraph{Podsystem skanowania biletów}
\begin{itemize}
\item takeScan()- cos jeszcze? ;p 
\end{itemize}


\paragraph{GUI}
\begin{itemize}
\item funkcje nasłuchujące akcji użytkownika, zaktualizuj tekst tu zmien kolor tam, itp.
\end{itemize}

\paragraph{Presenter}
rozdzielanie pracy na podsystemy, wywołuje funkcje z innych zależnych podsystemów
\begin{itemize}
\item UpdateLocalData()
\item UpdateRemoteData()
\item ManageScanning()
\end{itemize}

\paragraph{Podsystem zarządzania plikami}
\begin{itemize}
\item WriteUserData/ReadUserData
\begin{itemize}
\item WriteFriendsList()
\item ReadFriendsList()
\item WriteVisitedPlaces() - z ocenami
\item ReadVisitedPlaces()
\item WriteUserInfo() - login, e-mail, itp.
\item ReadUserInfo()
\end{itemize}
\item WriteTravelPlan()
\item ReadTravelPlan()
\item WriteDayPlan()
\item ReadDayPlan()
\end{itemize}

\paragraph{Podsystem kontaktu serwera z aplikacją}
\begin{itemize}
\item InitializeConnection()
\item Disconnect()
\item SendUserData()
\item SendTravelData()
\item SendDayPrograme()
\item SendLocalization()
\item UpdateChanges() – nie wiem czy nie za ogólnie
\item ChargeTravel() – jak nie będziemy mieć połączenia z Internetem
\item SendPreferences()-może być w SendUserData ? 

\end{itemize}

\paragraph{Podsystem kontaktu z serwerem}
\begin{itemize}
\item InitializeConnection()
\item Disconnect()
\item LoadDataFromGoogle()
\item LoadDataFromHere()
\item UpdateUserData()
\item UpdateTravelData()
\item UpdateDayProgram()
\item SendMaps()
\item SetUserData() – dane osobowe mail, imie,nazwisko,numer telefonu,znajomi itp
\item SetTravelData()
\item SetDayPrograme()
\item SetLocalization()

\end{itemize}

\paragraph{Podsystem polecający}
\begin{itemize}
\item ChargeMarksFromGoogle() – czy to  może jest w  LoadDataFromGoogle()
\item DetermineHierarchyOfPreferences() –na jakich typach atrakcji najbardziej użytkownikowi zależy
\item FindPreferencesInGoogle() – jeśli użytkownik jeszcze nie ma preferencji to wybieray te najbardziej polecane z Google
\item FindUserPreferences() – jeśli użytkownik ma preferencje to znajdujemy miejsca, które najbardziej mu odpowiadają
\item SetPreferences(Location) – użytkownik ocenia daną lokację jednocześnie ustawiając preferencję ( podnosi w hierarchii dane kategorie – słowa klucze np. muzeum)
\item FindBestEquivalent()- znajduje jak najbardziej zbliżoną kategoriami do danej atrakcji lokalizację – takie coś jak zobacz też, inni którzy tam byli ocenili wysoko także takie miejsca

\end{itemize}

\end{document}
