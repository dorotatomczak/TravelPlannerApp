\documentclass[10pt,twoside,a4paper]{report}

\usepackage{polski}
\usepackage{geometry}
\usepackage{indentfirst}
\usepackage{sectsty}
\usepackage{helvet}
\usepackage{caption}
\usepackage{multirow}
\usepackage{makecell}
\usepackage{array}
\usepackage{subcaption}
\usepackage{graphicx}

\graphicspath{ {./images/} }

\chapternumberfont{\Large}
\chaptertitlefont{\huge}

\usepackage[font={normalsize,it}]{caption}

\linespread{1.5}

\usepackage[table]{xcolor}
\usepackage{fourier} 
\renewcommand\theadalign{bc}
\renewcommand\theadfont{\bfseries}

\newgeometry{tmargin=3cm, bmargin=3cm, lmargin=2.5cm, rmargin=2.5cm}

\begin{document}
\begin{center}
\par {Scenariusz tworzenia planu dnia} 
\end{center}

\par
Użytkownik widzi pusty ekran planu dnia.wpisuje miasto docelowe wyjazdu - Paryż. Wybiera przycisk "+" i wybiera atrakcje. Ukazuje się przed nim ekran wyszukiwania ze znakiem wyszukiwania,a po chwili z wynikami wyszukiwania w formie listy atrakcji. Wybiera jeden z elementów listy – katedra Notre Dame i czyta jej opis. Wybiera przycisk "Dodaj". Widzi ponownie ekran planu Dnia z dodaną przez siebie atrakcją. Wybiera w ten sposób kilka kolejnych atrakcji. Ponownie wybiera przycisk "+" i wybiera restauracje. Ponownie widzi ekran wyszukiwania z listą restauracji. Wybiera pasujący mu obiekt i przy użyciu przycisku "Dodaj" zostaje on dodany do  planu dnia który ponownie wyświetla się przed nim. Użytkownik wybiera przycisk "Ułóż" i czeka aż aplikacja zakończy obliczanie najbardziej optymalnej trasy. Po kilku sekundach plan dnia jest już gotowy. Użytkownik stwierdza jednak że potrzebuje w ciągu dnia odpoczynku dlatego postanawia przesunąć zwiedzanie katedry Notre Dame na następny dzień. Wybiera element z katedrą i przycisk "Przenieś", a następnie numer dnia podróży. Następnie stwierdza, że właściwie nie interesuje go sztuka sakralna, więc usuwa element z planu przesuwając go w prawo.
\end{document}
